%\RequirePackage{atbegshi}
\documentclass[french]{beamer}
\usepackage{etex}

\usepackage[beamer,utf8x]{preambuleTrm}



\usepackage{iwona}


\mode<presentation>
{
  %\usetheme{Bergen}%Montpellier,Madrid}
  % or ...Frankfurt, 
  \usetheme[secheader]{Montpellier}

  %\setbeamercovered{highly dynamic}
  % or whatever (possibly just delete it)
  \usefonttheme{progressbar}
%  \useoutertheme{progressbar}
  %\useinnertheme{progressbar}
 % \progressbaroptions{% titlepage=normal,
  %  imagename=/home/moi/Photos/Maths/prisoner
%}
}

%\usepackage{elephantbird}
\newcommand\MyBox[2][XXX]{%
  \fbox{\tabular[b]{C{#1}} #2 \endtabular}}

\setbeamertemplate{footline}[frame number]
\beamertemplatenavigationsymbolsempty

%\usepackage[french,vlined,boxed]{algorithm2e}
\usepackage{bookmark,multido}
%\usepackage{xlop}

\ifpdf \DeclareGraphicsRule{*}{mps}{*}{} \fi
\usepackage{tikz, pgfkeys, pgfopts, xstring}
\usetikzlibrary{automata,fit,trees,matrix,arrows,decorations.markings,shapes,arrows,chains,positioning,intersections,backgrounds,calc,through,mindmap,shadows,patterns, positioning}

\usepackage{tikz-uml}

\usepackage[underline=true,rounded corners=false]{pgf-umlsd}

\hypersetup{pdfpagemode=FullScreen}

\usepackage[detect-all]{siunitx}

\usepackage{caption}
\captionsetup{labelformat=empty,font=footnotesize}

\usepackage[square]{daedale}
\bibliographystyle{cyclope}


\usepackage{algo}

\renewcommand{\algocommentfont}{\tiny\ttfamily\itshape}
\renewcommand{\algokeywordfont}{\small\ttfamily}
\renewcommand{\algotextfont}{\footnotesize\ttfamily}
\newtheorem{exercice}{Exercice}


\setlength{\columnseprule}{0pt}

%\setlength{\parskip}{0pt}


\graphicspath{{/home/moi/IUT/Figures/}{/home/moi/VISA_III/figures_visa/}{/home/moi/Lycee/TDmaple/2006_7/}{/home/moi/Figures/Arbres_Graphes/}{/home/moi/Figures/FigSTI/}{/home/moi/Figures/FigMaple/}{/home/moi/Photos/Maths/}{/home/moi/Photos/Tehessin/}{/home/moi/Figures/FigSTI/}{/home/moi/Figures/FigSeconde/}{/home/moi/Lycee/Informatique/XCAS/2008_9/}{/home/moi/Lycee/TDmaple/2008_9/}{/home/moi/IUT/Thierry/Conversions/}{/home/moi/Photos/informathix/}{/home/moi/Lycee/Informatique/PafAlgo/}{/home/moi/Figures/figuresAlice/}{/home/moi/Figures/FigSeconde/}{/home/moi/Biocam/robot/}{/home/moi/Biocam/CoursIni3d/}{/home/moi/Biocam/central/figures/}{/home/moi/Biocam/central/}{/home/moi/IUT/INFO2/Images/}{/home/moi/IUT/INFO2/2011_12/graphes/}{/home/moi/IUT/INFO1/CM1/}{/home/moi/Figures/figuresTS/}{/home/moi/Photos/Tehessin/}{/home/moi/Figures/FigSTI/}{/home/moi/Alice/figures3/}{/home/moi/Alice/figures3/newcourbes/}{/home/moi/Alice/figures/}{/home/moi/Figures/figuresAlice/}{/home/moi/Photos/Maths/}{/home/moi/Lycee/TS/}{/home/moi/Lycee/TS/2009_10/}{/home/moi/Figures/FigMaple/}{/home/moi/Figures/figuresTS/newcourbes/}{/home/moi/Figures/FigEspace/}{/home/moi/PROG/HASKELL/FLOAT/}{/home/moi/LYCEE/LIL/4e/2016_17/Images/}{/home/moi/LYCEE/LIL/2nde/2016_17/Images/}{/home/moi/LYCEE/LIL/2nde/2016_17/fonctions/}{/home/moi/LYCEE/LIL/1S/2016_17/Images/}{/home/moi/LYCEE/LIL/1S/2016_17/Images/}{/home/moi/LYCEE/LIL/1S/2016_17/suites/suites1es/}{/home/moi/LYCEE/LIL/1S/2016_17/trigo/}{/home/moi/Photos/Maths/CAPTURES_TI/Suites/}{/home/moi/LYCEE/FIGURES/}{/home/moi/UCO/L1/ANGERS2018/Images/}}



\setcounter{tocdepth}{2} %\setcounter{page}{0}




\usepackage{fdsymbol}





\setbeamercolor{block title}{use=structure,fg=black,bg=blue!55!white}
\setbeamercolor{block body}{use=structure,fg=black,bg=blue!10!white}


\setbeamercolor{block title example}{use=structure,fg=black,bg=green!75!white}
\setbeamercolor{block body example}{use=structure,fg=black,bg=green!10!white}

\setbeamercolor{block title alert}{use=structure,fg=black,bg=red!75!white}
\setbeamercolor{block body alert}{use=structure,fg=black,bg=red!10!white}



\setbeamerfont{block body}{size*={10}{12}}
\setbeamerfont{block body alerted}{size*={10}{12}}
\setbeamerfont{block body example}{size*={10}{12}}


\newenvironment{exercise}{\begin{exampleblock}{Recherche}}{\end{exampleblock}}

\usepackage{scalefnt}
\setbeamertemplate{theorems}[normal size]


\usepackage{colortbl}

\begin{document}


\title[]% (optional, use only with long paper titles)
{Programmation linéaire en pratique}

\subtitle{L2 MIASHS}

\author[] % (optional, use only with lots of authors)
{Guillaume CONNAN }
% - Give the names in the same order as the appear in the paper.
% - Use the inst{?} command only if the authors have different
%   affiliation.

\institute{Université Catholique de l'Ouest - Rezé}% (optional, but mostly needed)

\logo{\includegraphics[width=0.1\textwidth]{logoUCO}}

%\logo{\includegraphics[scale=0.15]{big_connan}}

\date[] % (optional, should be abbreviation of conference name)
{2019}
% - Either use conference name or its abbreviation.
% - Not really informative to the audience, more for people (including
%   yourself) who are reading the slides online

\subject{ }



\beamerdefaultoverlayspecification{<+->}


\AtBeginSubsection[]
{\scriptsize
  \begin{frame}<beamer>
    \frametitle{Sommaire}
 {\tiny
\begin{multicols}{2}
    \tableofcontents[currentsection,currentsubsection]
       \end{multicols}
}

  \end{frame}
}





\AtBeginSection[]
{
  \begin{frame}<beamer>
    \frametitle{Sommaire}
 {\scriptsize
\begin{multicols}{2}
    \tableofcontents[currentsection]
       \end{multicols}
}

  \end{frame}
}


% If you wish to uncover everything in a step-wise fashion, uncomment
% the following command: 

\beamerdefaultoverlayspecification{<+->}




\newcommand{\TR}{\mathcal{T}}


\begin{frame}
%\ThisCenterWallPaper{1}{logoiremBigInv}
 \Large \titlepage 
\end{frame}

\begin{frame}
 \frametitle{Sommaire}
{\scriptsize
\begin{multicols}{2} 
 \tableofcontents
\end{multicols}
}

 
 \end{frame}

\normalsize


% début

\section{Programmation linéaire avec Sage}


\begin{frame}
  \begin{align*}
  \text{max : }&x_1+x_2+3x_3\\
  \text{tel que : }&x_1+2x_2 \leqslant 4\\
  &5x_3  - x_2 \leqslant 8 \\
  &x,y,z \geqslant 0.
\end{align*}
\end{frame}

\begin{frame}[fragile]
\begin{pythoncode}
sage: p = MixedIntegerLinearProgram()
sage: x = p.new_variable(real=True, nonnegative=True)
sage: p.set_objective(x[1] + x[2] + 3*x[3])
sage: p.add_constraint(x[1] + 2*x[2] <= 4)
sage: p.add_constraint(5*x[3] - x[2] <= 8)
\end{pythoncode}

\pause
  
\begin{pythoncode}
sage: p.solve()
8.8
sage: p.get_values(x)
{1: 4.0, 2: 0.0, 3: 1.6}
\end{pythoncode}




\end{frame}


\section{Exemples de référence}


\subsection{Variables à indices simples}

\begin{frame}\frametitle{Exemple 1}

  Une société de jouets produit des trains, des camions et des voitures, en utilisant 3 machines.
Les disponibilités quotidiennes des 3 machines  sont 430, 460 et 420 minutes, et
les profits par train, camion et voiture sont respectivement 3€, 2€ et 5€. 
Les temps nécessaires sur chaque machine sont :

\begin{center}
\begin{tabular}{|c|c|c|c|}
\hline
 Temps Machines&Train&Camion&Voiture \\
\hline
$M_1$  &1&2&1 \\
\hline
$M_2$  &3&0&2 \\
\hline
$M_3$  &1&4&0 \\
\hline
\end{tabular}
\end{center}

L’objectif est de déterminer le plan de production réalisant un profit total maximal.
\end{frame}




\begin{frame}[fragile]
\begin{pythoncode}
sage: p = MixedIntegerLinearProgram(maximization=True)
sage: x = p.new_variable(nonnegative=True)
sage: p.set_objective(3*x[1] + 2*x[2] + 5*x[3])
sage: p.add_constraint(x[1] + 2*x[2] + x[3] <= 430)
sage: p.add_constraint(3*x[1] + 0*x[2] + 2*x[3] <= 460)
sage: p.add_constraint(1*x[1] + 4*x[2] + 0*x[3] <= 420)
\end{pythoncode}

  \pause

  \begin{pythoncode}
sage: p.solve()
1350.0
sage: p.get_values(x)
{1: 0.0, 2: 100.0, 3: 230.0}
\end{pythoncode}




  
\end{frame}




\begin{frame}[fragile]
\begin{pythoncode}
sage: p.add_constraint(x[1]>=10)
sage: p.solve()
1310.0
sage: p.get_values(x)
{1: 10.0, 2: 102.5, 3: 215.0}
\end{pythoncode}
\end{frame}


\begin{frame}[fragile]
\begin{pythoncode}
sage: x = p.new_variable(nonnegative=True, integer=True)

...

sage: p.solve()
1309.0
sage: p.get_values(x)
{1: 10.0, 2: 102.0, 3: 215.0}
\end{pythoncode}


\end{frame}



\subsection{Variables à indices multiples}


\begin{frame}\frametitle{Exemple 2}

  Trois machines $M_1$, $M_2$, et $M_3$ peuvent  produire chacune deux types de pièces $P_1$
  et $P_2$. Le temps de fabrication d’une pièce $P_i$ sur la machine $M_j$ est donné dans
  le tableau suivant (en heures) :

\begin{center}  
\begin{tabular}{|c|c|c|c|}
 \cline{2-4}
     \multicolumn{1}{c|}{}
  &$M_1$&$M_2$&$M_3$ \\
\hline
 $P_1$ &3&4&4 \\
\hline
 $P_2$ &4&6&5 \\
\hline
\end{tabular}
\end{center}
\pause

On veut fabriquer au moindre coût 6 pièces de type $P_1$ et 8 pièces de type $P_2$. La
machine  $M_1$ est  disponible 14  heures les  machines $M_2$  et $M_3$  sont disponibles
chacune 24  heures. Le  coût horaire  de $M_1$  (respectivement $M_2$  et $M_3$)  vaut 7€
(respectivement 5€ et 6€). 


\end{frame}





\begin{frame}
  
\begin{center}  
\begin{tabular}{|c|c|c|c|}
 \cline{2-4}
     \multicolumn{1}{c|}{}
  &$M_1$&$M_2$&$M_3$ \\
\hline
 $P_1$ &3&4&4 \\
\hline
 $P_2$ &4&6&5 \\
\hline
\end{tabular}
\end{center}

  
  \begin{itemize}
  \item On  note $x_{ij}$ le nombre  d'unités de $P_i$ fabriqués  par la machine
    $M_j$
  \item Le temps de fabrication sur $M_1$ est donc $3x_{11}+4x_{21}$
  \item Le coût correspondant est donc $7(3x_{11}+4x_{21})$
    \item    La    machine   $M_1$    doit    travailler    au   maximum    14h:
      $3x_{11}+4x_{21}\leqslant 14$
    \item Il faut au moins 6 pièces $P_1$: $x_{11}+x_{12}+x_{13}\geqslant 6$
  \end{itemize}
\end{frame}



\begin{frame}[fragile]
\begin{pythoncode}
sage: p = MixedIntegerLinearProgram(maximization=False)
sage: x = p.new_variable(nonnegative=True, integer=True)
sage: p.set_objective(7*(3*x[1,1] + 4*x[2,1]) + 5*(4*x[1,2] + 6*x[2,2]) + 6*(4*x[1,3] + 5*x[2,3]))
sage: p.add_constraint(3*x[1,1] + 4*x[2,1] <= 14)
sage: p.add_constraint(4*x[1,2] + 6*x[2,2] <= 24)
sage: p.add_constraint(4*x[1,3] + 5*x[2,3] <= 24)
sage: p.add_constraint(p.sum(x[1,j] for j in range(1,4)) >= 6)
sage: p.add_constraint(p.sum(x[2,j] for j in range(1,4)) >= 8)
\end{pythoncode}

  \pause

\begin{pythoncode}
sage: p.solve()
362.0
sage: p.get_values(x)
{(1, 1): 2.0, (1, 2): 3.0, (1, 3): 1.0, (2, 1): 2.0, (2, 2): 2.0, (2, 3): 4.0}
\end{pythoncode}


  

\end{frame}


\begin{frame}[fragile]
\begin{Verbatim}
sage: ?p.sum
Docstring:     
   Efficiently computes the sum of a sequence of "LinearFunction"
   elements

   Note: The use of the regular "sum" function is not recommended as
     it is much less efficient than this one

   EXAMPLES:

      sage: p = MixedIntegerLinearProgram()
      sage: v = p.new_variable(nonnegative=True)

   The following command:

      sage: s = p.sum(v[i] for i in range(90))

   is much more efficient than:

      sage: s = sum(v[i] for i in range(90))
\end{Verbatim}
\end{frame}


\subsection{Problème du sac à dos}


\begin{frame}
  \begin{itemize}
  \item 
  Le problème dit  du \og sac à dos  \fg\ est le suivant. Nous avons  en face de
  nous  une   série  d'objets   ayant  tous  un   poids  propre,   ainsi  qu'une
  \og{}utilité\fg\  mesurée  par un  réel.  Nous  souhaitons maintenant  choisir
  certains de ces  objets en nous assurant  que la charge totale  ne dépasse pas
  une  constante  $C$,  la meilleure  fa\c  con  de  le  faire étant  pour  nous
  d'optimiser la somme des utilités des objets contenus dans le sac.

\item
  
  Pour cela,  nous associerons à chaque  objet $o$ d'une liste  $L$ une variable
  binaire \texttt{prendre[o]}, valant 1 si l'objet doit être mis dans le sac, et
  0 sinon. Nous cherchons donc à résoudre le MILP suivant:

  \begin{itemize}
  \item $\text{max : }\sum_{o\in L}\text{utilité}_o\times \text{prendre}_o$
  \item $ \text{tel que : }\sum_{o\in L} \text{poids}_o\times \text{prendre}_o \leqslant C$
  \end{itemize}

\end{itemize}
\end{frame}

\begin{frame}[fragile]
\begin{pythoncode}
sage: C = 1
sage: L = ["Casserole", "Livre", "Couteau", "Gourde", "Lampe de poche"]
sage: p = [0.57,0.35,0.98,0.39,0.08]; u = [0.57,0.26,0.29,0.85,0.23]
sage: poids = {}; utilite = {}
sage: for o in L:
....:    poids[o] = p[L.index(o)]; utilite[o] = u[L.index(o)]
\end{pythoncode}

  \pause

\begin{pythoncode}
sage: p = MixedIntegerLinearProgram()
sage: prendre = p.new_variable( binary = True )
sage: p.add_constraint(
....:   p.sum( poids[o] * prendre[o] for o in L ) <= C )
sage: p.set_objective(
....:   p.sum( utilite[o] * prendre[o] for o in L ) )
sage: p.solve() 
1.42
sage: prendre = p.get_values(prendre)
\end{pythoncode}


  

\end{frame}


\begin{frame}[fragile]

  La solution trouvée vérifie bien la contrainte de poids:
  
\begin{pythoncode}
  sage: sum( poids[o] * prendre[o] for o in L )
  0.960000000000000
\end{pythoncode}


\pause

Faut-il prendre une gourde ?

\begin{pythoncode}
  sage: prendre["Gourde"]
  1.0
\end{pythoncode}


\pause

Quels objets prendre?

\begin{pythoncode}
sage: {o for o in L if prendre[o]}
{'Casserole', 'Gourde'}
\end{pythoncode}




\end{frame}

\section{Exercices}








\begin{frame}
  \begin{exercice}
    Une entreprise dispose de  deux usines de fabrications $U_1$ et  $U_2$ et de trois
    dépôts  $D_1$, $D_2$  et  $D_3$. Les  usines, qui  ont  des disponibilités  limitées,
    doivent  fournir  au  dépôt  les  quantités  demandées.  L’acheminement  des
    marchandises a un coût (frais de  transport, de carburant, taxes, etc) et ce
    coût varie suivant les destinations.

    Le  problème qui se pose  est celui de
    l’acheminement au moindre coût. 


    Les disponibilités de l’usine $U_1$ sont de 18 jours et celles 
de $U_2$ sont de 32 jours.

Les demandes des dépôts $D_1$, $D_2$ et $D_3$ sont respectivement de 9, 21 et
20 quantités.

Les coûts  de transports de l’usine $U_1$  vers chacun des dépôts  $D_1$, $D_2$  et  $D_3$
sont respectivement de 13,  9, 15, et ceux de l’usine  $U_2$ sont respectivement de
11, 10, et 18 par quantité transportée. 
  \end{exercice}
\end{frame}



\begin{frame}
  \begin{exercice}
  Une   compagnie   souhaite  minimiser   le   coût   de  transport   permettant
  d’approvisionner un  produit unique sur  5 dépôts de distribution  (clients) à
  partir de  2 usines de production.  Chaque usine a une  capacité de production
  limitée, et chaque client exprime une certaine demande en produit. Les données
  relatives à cette problématique sont résumées dans le tableau suivant:

  {\tiny 
    \begin{center}
  \begin{tabular}{|c|c|c|c|c|c|c|}
     \cline{2-7}
     \multicolumn{1}{c|}{}&Client 1&Client 2&Client 3&Client 4& Client 5& Capacité\\
    \hline
    Usine
    1&1,75\texteuro{}&2,25\texteuro{}&1,50\texteuro{}&2,00\texteuro{}&1,50\texteuro{}&60000\\
    \hline
    Usine
    2&2,00\texteuro{}&2,50\texteuro{}&2,50\texteuro{}&1,50\texteuro{}&1,00\texteuro{}&60000\\
    \hline
    Demande & 30000 &20000 &15000&32000&16000&\\
    \hline
  \end{tabular}
 \end{center}
}
\end{exercice}
\end{frame}



\begin{frame}
  \begin{exercice}
    La problématique  est identique au  cas précédent, mais considère  un réseau
    logistique faisant intervenir 2 niveaux:
    \begin{itemize}
    \item Niveau 1: Usines – Entrepôts
    \item Niveau 2: Entrepôts – Clients
    \end{itemize}
\pause
Les entrepôts  intervenant dans cette  organisation logistique ont  une capacité
limitée. Ils ont un rôle de massification  des flux. A noter que l’on s’autorise
éventuellement un  approvisionnement direct  d’un client  à partir  d’une usine,
sans transiter par un entrepôt.  
  \end{exercice}
\end{frame}



\begin{frame}
  \begin{exercice}
    Le problème calculatoire dénommé \emph{Subset  Sum} consiste à trouver, dans
    un ensemble d'entiers relatifs, un sous-ensemble non vide d'éléments dont la
    somme est nulle. Résoudre ce problème  avec un programme linéaire en nombres
    entiers sur l'ensemble  $\{28, 10, -89, 69,  42, -37, 76, 78,  -40, 92, -93,
    45\}$. 
  \end{exercice}
\end{frame}





\begin{frame}
  \begin{exercice}
    Vous devez assurer le fonctionnement d'un service hospitalier le week-end. Vos contraintes sont :			\begin{enumerate}
\item  il vous faut au moins 30 infirmières le vendredi, 20 le samedi et 12 le dimanche.
\item  Une personne travaille ou non, mais pas à temps partiel.		
\item  les infirmières peuvent travailler 1 ou 2 jours d'affilé mais pas les trois jours.
\item  Il n'y a pas de prise de service le dimanche.
\end{enumerate}

\pause


L’objectif est de ne monopoliser que le nombre minimal d’infirmières sur le week-end.
  \end{exercice}
\end{frame}

\begin{frame}
  \begin{exercice}
    Une  compagnie  pétrolière  utilise  des véhicules  compartimentés  pour  la
    distribution  de carburants  sur  les  stations service  de  son réseau.  On
    considère ici le cas où 3 types de carburants doivent être approvisionnés au
    moyen d’un camion citerne comportant 4 compartiments. Chaque client (station
    service)   exprime    son   besoin    en   volume    (hl)   par    type   de
    carburant.

    Contractuellement,  si une  demande n’est  pas satisfaite  par le
    pétrolier, une pénalité financière de 0,25€ est appliquée par hectolitre non
    livré, générant ainsi une perte financière pour la compagnie pétrolière.

    Le
    problème est de  déterminer le chargement optimal du  véhicule permettant de
    minimiser cette perte financière.  
  \end{exercice}
\end{frame}



\begin{frame}
  \begin{exercice}
    Un  groupe  informatique important  possédant  4  agences réparties  sur  le
    territoire  français doit  assurer  l’approvisionnement de  ses services  en
    clés USB.

    Les  besoins   prévisionnels  mensuels  par  agence   sont  connus.  3
    fournisseurs potentiels  sont susceptibles  de décrocher  tout ou  partie du
    contrat.

    Chacun  de  ces  fournisseurs  a fait  une  offre  relative  à  la
    fourniture de chacune  des agences (Coût = Coût produit  + Transport pour un
    lot de 1000 clés USB).

    Le problème  du responsable Achat est  de déterminer avec
    quel(s) fournisseur(s) il doit  contracter, et les flux fournisseurs-agences
    associés  de  manière à  minimiser  le  coût total  d’approvisionnement  des
    agences. 
  \end{exercice}
\end{frame}



\begin{frame}
  \begin{exercice}
    La problématique est similaire à  la précédente. Néanmoins, le Fournisseur 1
    n’est intéressé par le marché que pour un volume minimal de 15000 clés USB par
    mois. Quelle est la nouvelle politique d’achat optimale ? 
  \end{exercice}
\end{frame}

\begin{frame}
  \begin{exercice}
    Une  compagnie américaine  désire minimiser  le coût  d’acheminement de  ses
    productions de  ses usines  vers ses  centres de  distribution situés  à San
    Francisco, Denver, Chicago, Dallas, et New York.

    Chaque usine a une capacité
    de production limitée, et chaque centre de distribution exprime une certaine
    demande en produits  en regard de ses prévisions de  ventes. Cette compagnie
    exploite  des  usines  en  Caroline  du   Sud,  dans  le  Tennessee,  et  en
    Arizona.

    Elle envisage  la possibilité  d’implanter une  nouvelle unité  de
    production en  Arkansas d’une capacité  de 300, cette décision  induisant un
    coût fixe additionnel au coût de transport de 100k\$.

    Quelle  décision
    doit-elle prendre? 
  \end{exercice}
\end{frame}



\begin{frame}
  \begin{exercice}
    Une Groupe industriel exploite à l’heure  actuelle un réseau logistique de 5
    usines permettant l’approvisionnement de 4 entrepôts.

    Cette société souhaite réviser le  dimensionnement de cette organisation, et
    envisage la fermeture éventuelle d’une ou de plusieurs usines.

    L’effet direct  risque d’être une  augmentation du coût de  distribution qui
    pourrait néanmoins  être contrebalancé  par une  diminution globale  du coût
    d’exploitation du réseau.

    Quelle décision doit-elle prendre?
  \end{exercice}
\end{frame}



\begin{frame}
  \begin{exercice}
    Déterminer  comment investir  un excès  de trésorerie  de 400,000€  dans des
    placements à maturités 1,3  et 6 mois de manière à  maximiser le montant des
    intérêts acquis en respect d'un  volant de trésorerie utilisée mensuellement
    et une marge de sécurité mensuelle de 100,000€.

    Les caractéristiques des placements sont donnés dans le tableau ci-dessous:

    \begin{center}
    \begin{tabular}{|c|c|c|c|c|}
      \cline{2-5}
     \multicolumn{1}{c|}{}&Intérêt&Terme&Prix&Mois d'acquisition\\
            \hline
    1 mois&1\%&1&2000€&1, 2, 3, 4, 5 et 6\\
            \hline
      3 mois&4\%&3&3000€&1 et 4\\
            \hline
      6 mois&9\%&6&5000€&1 \\
           \hline
    \end{tabular}
  \end{center}
\end{exercice}
\end{frame}







\begin{frame}
  \begin{exercice}
    Un  loueur de  voitures possède  un  parc de  94 véhicules  répartis sur  10
    agences.

    Chaque agence  possède un effectif idéal permettant  de répondre au
    mieux aux demandes du marché.

    Cependant, la  réalité est  toute autre  et
    chaque agence possède soit un déficit, soit un excédent de voitures. Il faut
    donc rétablir l’équilibre dans chaque agence.

    Ayant le distancier donnant le
    kilométrage séparant chaque paire d’agences, et connaissant le coût moyen au
    km de  déplacement d’une voiture (1,3€),  il s’agit de minimiser  le prix de
    revient de ce rééquilibrage. 
  \end{exercice}
\end{frame}



\begin{frame}
   
%\begin{figure}
\begin{center}
  \includegraphics[width=\textwidth]{tableauRail}
\end{center}
%\end{figure}
\end{frame}

\begin{frame}
  \begin{exercice}
    {\footnotesize
    Un industriel  chimique doit  transporter 180  tonnes de  produits chimiques
    stockés  dans 4  entrepôts vers  trois centres  de retraitement.  Les moyens
    d’accès aux centres sont  la route et le rail, les  tarifs sont les suivants
    (les  premiers  chiffres concerne  le  transport  sur route,  les  deuxièmes
    concernent le  transport par rail. Le  sigle x indique que  le transport est
    impossible):

    {\tiny
    \begin{center}
      \begin{tabular}{|c|c|c|c|c|c|c|}
         \cline{2-7}
        \multicolumn{1}{c|}{}&
         \multicolumn{2}{c|}{$CT_1$}& \multicolumn{2}{c|}{$CT_2$}& \multicolumn{2}{c|}{$CT_3$}\\
        \cline{2-7}
                 \multicolumn{1}{c|}{}             &Route&Rail&Route&Rail&Route&Rail\\
        \hline
        $E_1$ (50t)&12&x&11&x&x&x \\
        \hline
        $E_2$ (40t)&14&12&x&x&x&x \\
        \hline
        $E_3$ (35t)&x&x&9&x&5&4 \\
        \hline
        $E_4$ (65t)&x&x&14&11&14&10 \\
        \hline
      \end{tabular}
    \end{center}
    }
    Par ailleurs, une contrainte est spécifiée au niveau du transport par rail :
    en  effet,  la quantité  transportée  doit  être  comprise  entre 10  et  50
    tonnes.\\
    Il n’y a pas de réglementations pour le transport par route.\\
    Comment
    acheminer 180 tonnes de produits chimiques au moindre coût?
    }
  \end{exercice}
\end{frame}


\begin{frame}
  \begin{exercice}
    Une  cargaison de  20 tonnes  doit être  transportée sur  un trajet  de cinq
    villes, avec  sur chaque  tronçon la possibilité  d’utiliser soit  la route,
    soit le train, soit l’avion. On peut  changer de mode de transport à chacune
    des villes traversées, mais le changement de mode a un coût.

    Le but est bien
    sûr de minimiser le coût de transport global.

{\footnotesize
    \smallskip
      \begin{tabular}{|c||*{4}{c|}}
        \hline
        \rowcolor{blue!35!white} Coût de transport&Villes 1-2&Villes 2-3&Villes 3-4&Villes 4-5\\
        \hline\hline
        \cellcolor{blue!35!white}Rail&30&40&30&60\\
        \hline
        \cellcolor{blue!35!white}Route&20&40&50&50\\
        \hline
        \cellcolor{blue!35!white}Air&40&10&60&40\\
        \hline
      \end{tabular}
      
\vspace{0.3cm}

\begin{tabular}{|c||*{3}{c|}}
      \hline
      \rowcolor{blue!35!white}\backslashbox{de}{vers}&Rail&Route&Air \\
      \hline\hline
      \cellcolor{blue!35!white}Rail&0&2&1 \\
      \hline
      \cellcolor{blue!35!white}Route&2&0&1 \\
      \hline
      \cellcolor{blue!35!white}Air&2&1&0 \\
      \hline
    \end{tabular}
}
\end{exercice}
\end{frame}





\begin{frame}
  \begin{exercice}
    Très souvent, les  problèmes rencontrés dans la  pratique professionnelle et
    en particulier  la logistique  sont assimilables à  des casse-têtes  plus ou
    moins élaborés. En voici un...
    
Partant d'une grille  carrée de taille 4×4 recouverte par  seize jetons, comment
enlever six jetons en laissant un nombre pair de 
jetons dans chaque ligne et chaque colonne de cette grille ?

Proposezune sortie graphique.

  \end{exercice}
\end{frame}

\end{document}
%%% Local Variables:
%%% mode: latex
%%% TeX-master: t
%%% End:
